\renewcommand{\FileId}{File: lab4.tex, Last changed: 2005-04-14}

%%%%%%%%%%%%%%%%%%%%%%%%%%%%%%%%%%%%%%%%%%%%%%%
% Analysis of Measured Loads
%%%%%%%%%%%%%%%%%%%%%%%%%%%%%%%%%%%%%%%%%%%%%%%

%\cleardoublepage
\chapter{Modelling of Measured Loads}

We want to model a measured load signal that changes characteristics
over time.
Suppose that a switching process would be an appropriate model for the
load.
Our strategy is then to fit a switching Markov chain of turning points
(SMCTP) to the measurements. This
model can then be used to calculate the expected rainflow matrix
for the load, or to simulate load processes.  The procedure is illustrated in the form of a flow chart in
Figure~\ref{FigFlowChart2}, which is explained and commented below.
\begin{figure}[htbp]
  % Figuren �r ritad i xfig och converteras med kommandot
  % fig2pstex fig/FigFlowChart2
  \begin{center}
    \resizebox{!}{!}{\input{fig/FigFlowChart2.pstex_t}}
  \end{center}
  \caption{\emph{Flow chart for the modelling of measurements of a load.}}
  \label{FigFlowChart2}
\end{figure}



%%%%%%%%%%%%%%%%%%%%%%%%%%%%%%%%%%%%%%%%%%%%%%%
% A Truck Load --- curves and straights
%%%%%%%%%%%%%%%%%%%%%%%%%%%%%%%%%%%%%%%%%%%%%%%

\section{Estimation from Time Signal}

In this example we have a measurement of a truck load, with the truck
driving on a road that consists of curves and straights with some
small bends, see~\cite[Example~6.3]{Johannesson99.PhD}.
The same piece of road has been measured twice.
The time signal is stored in the variable
\verb|x0| with time (seconds) in the first column and the load values
in the second column. Load and plot the data.
\begin{code}
>> load switchingload
>> plot(x0(:,1),x0(:,2))
\end{code}
The load changes
between two duty cycles which can be observed as changes in mean
and standard deviation.

%\begin{figure}[htbp]
%  \centerline{
%%    \resizebox{\figwidthA}{\figheightA}{\includegraphics{fig/FigExVolvoLV2_Load.eps}}
%  }
%  \caption{\emph{A measured truck load consisting of curves and straight
%      parts. Rainflow cycles with range smaller than the discretization step, $0.0429$, have been removed.}}
%  \label{FigEx3bVolvoLoad}
%\end{figure}

Does it seem appropriate to model the load
by a SMCTP model with two regime states?
%The load will be modelled %, following Figure~\ref{FigFlowChart2},
%by a time-reversible
%switching MCTP with two regime states.
The  two regime states represent straights
(regime 1, low
mean, large variation), and curves (regime 2, high mean, low variation).

First, we will remove small oscillations in the load, which
either come from measurement noise, or are irrelevant to the fatigue
damage. This is done by a rainflow filter, which deletes all rainflow cycles with ranges
smaller than a given threshold, which is often chosen as the
discretization step. Define the discretization and apply a rainflow filter:
\begin{code}
>> n = 32; param = [-1 1.2 n]; % Define discretization
>> u = levels(param);     % Discrete levels
>> delta = u(2)-u(1)      % Discretization step
>> TP = dat2tp(x0,delta); % Get turning points and rainflow filter
\end{code}
Compare the turning points of the original time signal with the
rainflow filtered turning points. How much does the rainflow filter
reduce the amount of data?
\begin{code}
>> TP0 = dat2tp(x0);
>> plot(x0(:,1),x0(:,2),TP0(:,1),TP0(:,2),TP(:,1),TP(:,2))
>> length(x0), length(TP0), length(TP)
\end{code}
Check the rainflow cycles and the damage before and after rainflow
filtering. Is there any significant difference? Try different values of
the damage exponent \verb|beta|.
\begin{code}
>> RFC0 = tp2rfc(TP0); subplot(1,2,1), ccplot(RFC0)
>> RFC = tp2rfc(TP);   subplot(1,2,2), ccplot(RFC)
>> beta = 6;                % Damage exponent
>> Dam0 = cc2dam(RFC0,beta) % Damage
>> Dam = cc2dam(RFC,beta)   % Damage, after rainflow filter
>> Dam/Dam0
\end{code}
%The rainflow filtering reduces the number of cycles by
%74 \%
%(from 1533 to 396), but less than $0.05~\%$ of the total damage was removed for
%$\beta>3$ ($10^{-3}~\%$ for $\beta=4$, and $10^{-6}~\%$ for
%$\beta=6$).

Examine the level crossing spectrum of the load by using \verb|tp2lc|, both for
the original sequence \verb|TP0| and for the rainflow filtered
sequence \verb|TP|.
Is there any difference between the two crossing spectra?
Is it possible, from the level crossings, to see that the load
consists of two subloads?

Examine if the load can be modelled as a Gaussian process by using
e.g.\ \verb|wnormplot|. Also compare the level crossing spectrum with
the one obtained from Rice's formula, see Computer
Exercise~\ref{lab1}.

Now we will discretize the load by using \verb|dat2dtp|, which makes
discretization to the nearest
discrete level.
\begin{code}
>> dtp = dat2dtp(param,TP,delta); % Discretized turning points
>> tp = [dtp(:,1) u(dtp(:,2))'];  % Discretized turning points
>> T = length(dtp);               % Number of turning points
>> clf, plot(x0(:,1),x0(:,2),TP(:,1),TP(:,2),tp(:,1),tp(:,2))
>> v=axis; hold on,
>> plot([v(1:2)],[u(2:end)'-delta/2 u(2:end)'-delta/2],'k:')
>> hold off, axis([v(1:2) param(1:2)])
\end{code}
What is the error
in damage due to the discretization?
\begin{code}
>> rfc = tp2rfc(tp);        % Get rainflow cycles
>> dam = cc2dam(rfc,beta)   % Damage, after discretization & rainflow filter
>> dam/Dam0
\end{code}
Next we will (by hand) identify the two duty cycles in the load
signal.
Find the times when load switches regime state, and split up the
load by using the routine \verb|splitload|. The first column of
\verb|tz| contains times (in seconds) and the second column contains the
regime state that the load switches to.
\begin{code}
>> tz = [2 2; 46.40 1; 114.5 2; 161 1; 225.1 2; 270 1; 337.5 2; ...
>> 384.8 1; 433.2 2; 600 1];
>> [xxd,xd,z] = splitload(dtp,tz);
>> plot(xxd{1}(:,1),xxd{1}(:,2))
>> plot(xxd{2}(:,1),xxd{2}(:,2))
>> hmmplot(xd(:,2),z,xd(:,1),[1 2],'','',1)
\end{code}
Here \verb|xxd{1}| and \verb|xxd{2}| contain subload 1 and 2,
respectively, \verb|xd| the switching load, and \verb|z| the regime
process.

%Then we identified the two duty cycles in the load signal,
% and calculated for each subload the
%min-max matrix and the max-min matrix.
%The discretization was made according to method 1 on
%page~\pageref{ChCycles:DiscMethod1}, discretization to the nearest
%discrete level, with $n=64$
%discretization levels, ranging from $u_1=-1.2$ to $u_n=1.5$.
%The min-max and max-min transition
%matrices were then obtained by using kernel
%smoothing and normalization.

\subsubsection*{Estimation of the Subloads}

%For simplicity we will
Now assume that the subloads are time-reversible, which
%from Section~\ref{SecMarkovTP:TimeReversibility}
implies that their expected
max-min matrices can be obtained from their respective expected min-max matrix through
$\bmh{G}=\bm{G}^T$.
For each subload we need to calculate the min-max and max-min
transition matrices
$\bm{Q}$ and $\bmh{Q}$, respectively. This can be done in the
following three steps:
\begin{enumerate}
\item \emph{Calculation of min-max matrix $\bm{F}$.} This is no
  problem if we have a measurement of  $\bm{F}$.
  (If instead we have measured the rainflow matrix $\bm{F}^{\rfc}$, it can be
  inverted to find the min-max matrix $\bm{F}$, see~\cite[Chapter~7]{Johannesson99.PhD}, and see also the routines
  \verb|rfm2mctp| and \verb|arfm2mctp|.)
  %Chapter~\ref{ChRainflowInversion}.
  In our case the subloads have been measured as time series and then it is
  straightforward to calculate the min-max matrix $\bm{F}$ (and
  max-min matrix $\bmh{F}$).
\begin{code}
>> dtp1 = dat2tp(xxd{1});
>> [mM1,Mm1] = tp2mm(dtp1);
>> F1 = dcc2cmat(mM1,n);
>> dtp2 = dat2tp(xxd{2});
>> [mM2,Mm2] = tp2mm(dtp2);
>> F2 = dcc2cmat(mM2,n);
>> cmatplot(u,u,{F1 F2})
\end{code}
\item \emph{Estimate $\bm{G}$ through smoothing.} To obtain an estimate of the expected min-max
 matrix $\bm{G}$, the min-max matrix $\bm{F}$ is
 smoothed using a 2-dimensional kernel smoother,
 see~\cite[Appendix~D]{Johannesson99.PhD}.
 %Appendix~\ref{AppKernel}.
\begin{code}
>> [G1s,h1] = smoothcmat(F1);
>> G1 = smoothcmat(F1,1,1.0,0);
>> [G2s,h2] = smoothcmat(F2);
>> G2 = smoothcmat(F2,1,0.8,0);
>> cmatplot(u,u,{G1s G2s; G1 G2})
>> cmatplot(u,u,{F1 F2; G1 G2})
\end{code}
Choose appropriate values of the smoothing parameter \verb|h|.
 (Here we can improve the estimates, if the max-min matrix
 $\bmh{F}$ is also used, by smoothing the sum
 $\bm{F}+\bmh{F}^T$, instead of smoothing only $\bm{F}$.)

\item \emph{Normalizing.} The min-max and max-min transition matrices
  $\bm{Q}$ and $\bmh{Q}$, respectively, are obtained from the expected min-max
  and max-min matrices $\bm{G}$ and
  $\bmh{G}=\bm{G}^T$, respectively, by normalizing each
  row sum to 1. This is done automatically by the programs
  (\verb|mctp2rfm| and \verb|smctp2rfm|).
\end{enumerate}


\subsubsection*{Estimation of the Regime Process}



The $\bm{P}$-matrix for the regime process is obtained through
ML-estimation
\begin{equation}
  \bm{P}^* = \left(
  \begin{array}{cc}
    1-p^*_{12} & p^*_{12} \\
    p^*_{21} & 1-p^*_{21}
  \end{array} \right)
\end{equation}
where
\begin{eqnarray}
  p^*_{12}
  &=& \frac{N_{12}}{N_1}=
  \frac{\#\{\mbox{Jumps from 1 to 2}\}}{N_1} =
  \frac{4}{N_1} = 0.0134, \\
  p^*_{21} &=& \frac{N_{21}}{N_2}=\frac{\#\{\mbox{Jumps from 2 to 1}\}}{N_2} =
  \frac{4}{N_2} = 0.0081 .
\end{eqnarray}
where $N_{ij}$ is the number of switches from regime state $i$ to
state $j$, and $N_i$
is the total number of turning points in regime state $i$.
Now we estimate \verb|P| from the rainflow filtered and discretized
load signal.
\begin{code}
>> N1 = length(dtp1), N2 = length(dtp2)
>> N12 = 4; N21 = 4;
>> p1=N12/N1; p2=N21/N2;
>> P = [1-p1 p1; p2 1-p2]  % P-matrix
>> statP = mc2stat(P)      % Stationary distribution
\end{code}

From the estimated SMCTP model (\verb|P|, \verb|G1|, and \verb|G2|), we can calculated the expected
rainflow matrix
%which is shown in Figures~\ref{FigEx3bRFCint}a,b.
\begin{code}
>> GG = {G1 []; G2 []};
>> [Grfc,mu_rfc] = smctp2rfm(P,GG);
>> cocc(param,RFC,Grfc)
>> Frfc = dtp2rfm(dtp(:,2),n);
>> cmatplot(u,u,{Frfc Grfc*T/2})
\end{code}
Calculate the damage and the damage matrix.
\begin{code}
>> beta = 6;
>> Dam0, Dam, dam
>> damG = cmat2dam(param,Grfc,beta)*T/2
>> damG/Dam0
>> beta = 4;
>> Dmat = cmat2dmat(param,Frfc,beta);
>> DmatG = cmat2dmat(param,Grfc,beta)*T/2;
>> cmatplot(u,u,{Dmat,DmatG},3)
\end{code}
Simulate the estimated SMCTP model and compare it with the measured signal.
\begin{code}
>> [xsim,zsim] = smctpsim(P,GG,T);
>> figure(1), hmmplot(u(xd(:,2))',z,1:length(xd),[1 2],'','',1)
>> figure(2), hmmplot(u(xsim)',zsim,1:T,[1 2],'','',1)
\end{code}
Make some more simulations and compare the simulated load signals.


%\begin{figure}[htbp]
%  \begin{tabular}{cc}
%    {\small (a) Rainflow matrix}  &
%    {\small (b) Rainflow matrix} \\
%    \resizebox{\figwidthB}{\figheightBB}{\includegraphics{fig/FigExVolvoLV2_RFMs.eps}} &
%    \resizebox{\figwidthB}{!}{\includegraphics{fig/FigExVolvoLV2_RFMsCC.eps}}\\
%    {\small (c) Crossing intensity, $\mu(u)$}  &
%    {\small (d) Damage} \\
%    \resizebox{\figwidthB}{!}{\includegraphics{fig/FigExVolvoLV2_Cross1.eps}} &
%     \resizebox{\figwidthB}{!}{\includegraphics{fig/FigExVolvoLV2_Dam.eps}}
%  \end{tabular}
%  \caption{\emph{Example~\ref{Ex3b}.
%      (a) 3D-plot of the expected rainflow matrix, $\bm{G}^{\rfc}$.
%      (b) Iso-lines of the expected rainflow matrix, $\bm{G}_T^{\rfc}$, together with rainflow cycles found in the load.
%      (c) Level upcrossing intensity, $\mu(u)$, compared with observed
%      upcrossing spectrum.
%      (d) Relative difference of damages,
%      $(\Dam_{\beta}(\bm{G}_T^{\rfc})-\Dam_{\beta}(X_k))/
%      \Dam_{\beta}(\bm{G}_T^{\rfc})$;
%      the mean \Solid,
%      the median \Dashdot,
%      the dashed lines \Dashed\ are the quantiles 0.1, 0.3, 0.7, 0.9,
%      obtained from 500 simulations of the estimated SMCTP model,
%      the \texttt{*}-line is the measured load, and
%      the $\Box$-line is the discretized measured load.
%      }}
%      \label{FigEx3bRFCint}
%\end{figure}


%%%%%%%%%%%%%%%%%%%%%%%%%%%%%%%%%%%%%%%%%%%%%%%
% A Truck Load --- curves and straights
%%%%%%%%%%%%%%%%%%%%%%%%%%%%%%%%%%%%%%%%%%%%%%%

\section{Decomposition of a Mixed Rainflow Matrix}



The goal of this example is to make a decomposition of the measured mixed
rainflow matrix, see~\cite[Example~8.1]{Johannesson99.PhD}.
The measurement is is the same  truck load as before.
We will make the decomposition with different assumptions on the
parametrization of the subloads, according to scenarios 1 and 3, see Section~\ref{lab2:Decomposition}.

In this example we will discretize the measured load by $n=16$ discrete
levels, ranging from $u_1=-1.0$ to $u_n=1.2$.
The pre-processing of the time signal is performed in the same way as previously.
\begin{code}
>> n = 16; param = [-1 1.2 n]; % Define discretization
>> u = levels(param);     % Discrete levels
>> delta = u(2)-u(1)      % Discretization step
>> TP = dat2tp(x0,delta); % Get turning points and rainflow filter
\end{code}
Check the difference in damage between the original load history and
the rainflow filtered one.

Next we will discretize the load and compute its damage.
\begin{code}
>> dtp = dat2dtp(param,TP,delta); % Discretized turning points
>> tp = [dtp(:,1) u(dtp(:,2))'];  % Discretized turning points
>> T = length(dtp);               % Number of turning points
>> rfc = tp2rfc(tp);        % Get rainflow cycles
>> beta = 6;
>> dam = cc2dam(rfc,beta)   % Damage, after discretization & rainflow filter
>> dam/Dam0
\end{code}
From the discretized load we compute the observed rainflow
matrix, which will be the input to the decomposition.
\begin{code}
>> Frfc = dtp2rfm(dtp(:,2),n);  % Observed rainflow matrix
\end{code}

For scenario 1, the min-max matrices \verb|G1| and \verb|G2| for the subloads
will be treated as a priori information, and will hence be considered
known. Therefore, for scenario 1, we only need to estimate the
$\bm{P}$-matrix with two parameters $p_1$ and $p_2$.
To obtain \verb|G1| and \verb|G2| we split the time signal,
and estimate them (and then forget about the time signal).
\begin{code}
>> tz = [2 2; 46.40 1; 114.5 2; 161 1; 225.1 2; 270 1; 337.5 2; ...
>> 384.8 1; 433.2 2; 600 1];
>> [xxd,xd,z] = splitload(dtp,tz);
>> hmmplot(xd(:,2),z,xd(:,1),[1 2],'','',1)
>> dtp1 = dat2tp(xxd{1});
>> [mM1,Mm1] = tp2mm(dtp1);
>> F1 = dcc2cmat(mM1,n);
>> G1 = smoothcmat(F1,1,1.0,0);
>> dtp2 = dat2tp(xxd{2});
>> [mM2,Mm2] = tp2mm(dtp2);
>> F2 = dcc2cmat(mM2,n);
>> G2 = smoothcmat(F2,1,0.8,0);
\end{code}
Plot the estimated matrices \verb|G1| and \verb|G2|.

By using the observed rainflow matrix \verb|Frfc| (and \verb|G1|
and \verb|G2|) we do the decomposition.
\begin{code}
>> known1.F = {G1 []; G2 []};   % known min-max and max-min matrices
>> init1.P = P;                 % initial guess of P-matrix
>> warning off                  % Don't display warnings
>> [Fest1,Est1] = estsmctp(Frfc,'P','ML',known1,[],init1);
>> Est1.P          % Estimated P-matrix
>> mc2stat(Est1.P) % Estimated stationary distribution
\end{code}

For scenario 3, we will for each subload use the simple parametric
model, which we used in Computer Exercise~\ref{lab2}.
Hence, for scenario 3, we have 10 parameters to estimate: $p_1$, $p_2$, and 4
parameters for each subload, \\
$\bm{\theta}=(p_1,~p_2,~x_{11}~,x_{21}~,s_1~,\lambda_1,~x_{12}~,x_{22}~,s_2~,\lambda_2)$.
\begin{code}
>> known3.Ffun = 'f_funm';      % Function for calculating a submodel
>> known3.trModel2X = 'tr_m2x'; % transform from Model to X-vector
>> known3.trX2Model = 'tr_x2m'; % transform from X-vector to model
>> known3.param = param;
>> init3.P = P;       % initial guess of P-matrix
>> M1.x0=[0.1 0.1]; M1.s=0.15; M1.lam=2; % submodel 1
>> M2.x0=[0.5 0.7]; M2.s=0.1; M2.lam=1;   % submodel 2
>> init3.M = {M1 M2}; % initial guess of Models for min-max matrices
\end{code}
To shorten the computation time you can lower the accuracy by setting the
input \verb|OPTIONS| which is used by \verb|fmins|. See help
\verb|options| for more details.
\begin{code}
>> OPTIONS(2)  = 1e-1;    % the termination tolerance for x;
>> OPTIONS(3)  = 1e-1;    % the termination tolerance for F(x);
>> [Fest3,Est3] = estsmctp(Frfc,'P,CalcF','ML',known3,[],init3);
>> Est3.P          % Estimated P-matrix
>> mc2stat(Est3.P) % Estimated stationary distribution
>> Est3.M{:}       % Estimated parameters in models
\end{code}

Compare the results (\verb|Fest1|, and \verb|Fest2|) in
terms of estimated $\bm{P}$-matrices and there
stationary distributions. Remember that the model estimated from the
time signal has transition
matrix \verb|P| and stationary distribution \verb|statP|.
Hopefully, you can observe that even though the estimates of the parameters $p_1$
and $p_2$
differs from \verb|P|, the estimates of the stationary
distribution are quite accurate.

For each estimated SMCTP model we can compute its expected damage
as a function of the damage exponent $\beta$.
We can also compute the damage from the measured time signal.
\begin{code}
>> beta = 3:0.2:8;
>> Dam0 = cmat2dam(param,Frfc,beta)/(T/2); % Damage from load signal
>> FrfcEst1 = smctp2rfm(Fest1.P,Fest1.F);
>> Dam1 = cmat2dam(param,FrfcEst1,beta);   % Damage, scenario 1
>> FrfcEst3 = smctp2rfm(Fest3.P,Fest3.F);
>> Dam3 = cmat2dam(param,FrfcEst3,beta);   % Damage, scenario 3
>> plot(beta,Dam0,'b',beta,Dam1,'r',beta,Dam3,'g')
>> plot(beta,Dam1./Dam0,'r',beta,Dam3./Dam0,'g')
\end{code}
Are the results, in terms of damage, acceptable?
Do they agree better for small or for large values of the damage exponent $\beta$.
