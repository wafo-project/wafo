\cleardoublepage
\pagenumbering{roman}
\chapter*{WAFO and module ``lagrange''}
\addcontentsline{toc}{chapter}{WAFO and module Lagrange}
\vspace{-5mm}

\subsubsection*{On the module ``lagrange''}
This is the 2017 version of a tutorial for how to use the \ML{} {\sc Wafo}-module 
{\tt lagrange} for analysis and simulation of random Lagrange waves, 
which is included in the {\sc Wafo} toolbox in the folder {\tt lagrange}. 
The module consists of a number of \ML{} m-files and it requires a
standard \ML{} setup together with the {\sc Wafo} 2017 toolbox. 
In some example we use routines from other \ML{} toolboxes, 
like the {\tt signal} toolbox. 

The {\sc Wafo} module {\tt lagrange}  contains routines for generation of 2D and 3D 
Gauss-Lagrange waves where a Gaussian process for the vertical movements of water particles is linked with  correlated Gaussian horizontal movements. This is the \fo model.  
The module also contains routines for \so order 3D Stokes-Lagnage waves. 
These routines are based on \ML{} and Fortran routines written by Marc Prevosto, IFREMER, Brest, France. The routines have been adapted to work together with {\sc Wafo}, including an option to use the \ML{} Parallel Computing toolbox.
 
 The \ML{} code used for the examples in this tutorial can be found in three 
 script files \verb+WafoLChx.m+. Some editing has been made on figures, 
 and some simulations have been run with more replicates than in 
 \verb+WafoLChx.m+.\footnote{The {\sc Wafo} lagrange routines were 
 originally published  as a stand-alone package {\sc Wafo}L, and we keep 
 that nomenclature thoughout this tutorial.} For help on the module, write 
 {\tt help lagrange}.

The routine \verb+spec2ldat3DP+ is a version of \verb+spec2ldat3DM+ 
adapted for parallel processing with the Parallel Computing Toolbox in \ML{}. 

Valuable comments on the tutorial and the use of {\sc Wafo} and 
{\tt lagrange} by several users all over the world  are gratefully acknowledged. 
Comments on {\sc Wafo}, the {\tt lagrange} tutorial, and the 
routines are appreciated to 
%
%\noindent
\verb+wafo@maths.lth.se+

\subsubsection*{On {\sc Wafo}}

\progname{} is built of modules of platform independent \ML{} m-files
and a set of executable files from \verb!C++! and \verb+Fortran+
source files. These executables are platform and \ML{-version} dependent,
and they have been tested with recent \ML{} and {\sc Windows} installations.
The latest version can be downloaded from

\verb+https://github.com/wafo-project+,

\noindent where you also find {\sc Pywafo}, a Python version. 


\smallskip
Older versions of the toolbox can be downloaded
from the \progname{} homepage

\noindent
\verb+http://www.maths.lth.se/matstat/wafo/+

\smallskip
\noindent
There you can also find links to exercises and articles using \progname{},
and notes about its history.
For help on the toolbox, write \verb+help wafo+. 


\subsubsection*{
The owners of the {\sc Wafo} package are}

\noindent Per Andreas Brodtkorb: \verb+per.andreas.brodtkorb@gmail.com+

\noindent Georg Lindgren: \verb+georg.lindgren@gmail.com+

\noindent Igor Rychlik: \verb+igor.rychlik@gmail.com+

\noindent New contributor: \verb+Marc.Prevosto@ifremer.fr+


